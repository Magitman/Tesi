\chapter{Conclusioni}
\label{chap:Conclusioni}


Nei capitoli precedenti sono state illustrate ed analizzate le tecnologie che hanno reso possibile l'implementazione di un servizio in grado di validare codici bancari, comunemente chiamati IBAN. Questo lavoro si è basato su una solida architettura e una struttura software ben progettata, che hanno permesso, anche grazie al supporto dei più recenti framework, la creazione di un'infrastruttura affidabile ed efficiente, garantendo un risultato sicuro.\\
Per soddisfare appieno le esigenze dell'azienda cliente e fornire un servizio all'avanguardia, è stato necessario uno studio approfondito delle tecnologie e degli strumenti presentati nel primo capitolo. La configurazione avanzata e le tempistiche sfidanti hanno reso cruciale la fase di pianificazione. Infatti, l'analisi dettagliata del caso d'uso principale del servizio ha portato alla creazione di un Diagramma di Sequenza di Sistema [figura \ref{fig:Architettura.png}], che ha chiarito gli eventi di input e di output relativi al processo e le interazioni tra i vari livelli architetturali.\\
Va comunque precisato che lo sviluppo del servizio è un processo in continua evoluzione. In particolare, i servizi REST sono caratterizzati da una mappatura delle richieste HTTP in entrata differenziate in base alla versione di ogni metodo esposto. Nel caso attuale, la versione del servizio è "1.0", pertanto rappresenta solo l'inizio di un progetto che sarà soggetto a futuri aggiornamenti ed espansioni. Una delle modifiche in avvenire già discusse riguarda la sostituzione della chiamata all'API esterna con una chiamata a un database locale del gruppo bancario, al fine di migliorare sia la sicurezza che la centralità del servizio. Altre implementazioni future comprenderanno l'autenticazione degli utenti o dei dipendenti che richiamano il servizio, attraverso le loro credenziali effettive, ampliando la comunicazione con più tabelle del database per autenticare gli utenti e memorizzare le informazioni relative alle richieste.\\
Posso dunque affermare che gli obiettivi stabiliti all'inizio del percorso di stage sono stati raggiunti con successo, permettendo di migliorare le mie competenze nello sviluppo di servizi web e nell'uso delle tecnologie e dei protocolli più avanzati. Inoltre, la collaborazione con un team multidisciplinare e le sfide affrontate durante lo sviluppo mi hanno dato la possibilità di accrescere le capacità di lavoro di squadra, risoluzione dei problemi e rispetto delle scadenze. Il percorso di sviluppo affrontato ha rappresentato per me una preziosa occasione di crescita personale e professionale. Guardando al futuro, il servizio di validazione IBAN esposto in questo elaborato sarà una risorsa in continua evoluzione, pronta a soddisfare nuove esigenze e sfide nel mondo della validazione dei codici bancari.
