\chapter{Introduzione}
\label{chap:Introduzione}



La presente tesi descrive l'attività svolta durante il periodo di stage curricolare presso l'azienda multinazionale Alten, con sede a Torino. L'obiettivo centrale di questo studio è stato l'elaborazione di una soluzione dedicata alla verifica dei codici IBAN per un rinomato gruppo bancario di rilevanza europea. In tal senso, è stato ideato e sviluppato un applicativo volto a consentire all'utente di inserire il proprio codice IBAN per ottenere immediatamente una conferma circa la sua validità. Questo servizio è stato progettato con l'intento di costituire una risorsa di fondamentale importanza sia per la clientela retail, comprendente clienti individuali e persone fisiche, sia per il personale interno dell'istituto bancario.\\
Data la costante crescita dell'utenza bancaria e l'espansione continua dei servizi offerti, il numero di clienti all'interno del gruppo bancario è in costante aumento, una tendenza che si prevede proseguirà anche nei prossimi anni. Pertanto, l'accesso a un servizio semplice, intuitivo e soprattutto affidabile per la verifica e la validazione degli ID bancari è diventato fondamentale per tutte le operazioni relative ai conti correnti. Questa soluzione garantisce sia agli utenti sia agli operatori la possibilità di operare in modo sicuro e tempestivo.\\
Un aspetto altrettanto cruciale del progetto è stato il tracciamento delle operazioni eseguite attraverso il servizio, in modo da poter registrare le richieste, lo stato dell’elaborazione e le risposte correlate a ogni interazione. Questo requisito ha comportato l'implementazione di un database che è stato concepito per operare “in-memory”, ossia per archiviare e gestire i dati direttamente nella memoria principale del sistema: il vantaggio quindi consiste nel fatto che i dati vengano mantenuti nella RAM del computer, garantendo un accesso significativamente più rapido rispetto ai tradizionali database su disco.\\
Inoltre, poiché l’applicativo è reso disponibile anche ai clienti del gruppo bancario, la sua realizzazione ha richiesto l'utilizzo di linguaggi e tecnologie moderne e all'avanguardia. L'infrastruttura di back-end è stata sviluppata utilizzando il framework Spring Boot, il quale ha semplificato e al contempo ottimizzato l'implementazione del servizio REST. Infine, è stato utilizzato Spring Data JPA per la creazione e la gestione del database. Con “servizio REST” (dall’inglese REST service) si fa riferimento ad un tipo di architettura software utilizzata per la progettazione e l’esposizione di interfacce web che consentono alle applicazioni di comunicare e trasferire dati tramite il protocollo HTTP.\\
Tuttavia, il progetto non è limitato alla mera realizzazione del servizio back-end. È stato altresì necessario sviluppare un punto di accesso intuitivo e familiare sia per i clienti che per gli operatori: una pagina web. Con il vincolo di rispettare criteri di sicurezza, affidabilità, prestazioni e aspetto visivo, è stato adottato il framework open source Angular, che si presta alla creazione di applicazioni web dinamiche e complesse. La natura versatile di Angular lo rende particolarmente adatto allo sviluppo di applicazioni aziendali e di gestione dati, risultando un'opzione ideale per il servizio in questione.\\
Il secondo capitolo si focalizza sulle tecnologie utilizzate per la realizzazione dell’applicativo, ossia vengono presentati in maniera dettagliata i linguaggi di programmazione e i framework utilizzati sia per la realizzazione del comparto back-end, sia per quello front-end. Ci si sofferma, inoltre, sulle tecnologie più rilevanti nell'ambito dello sviluppo, senza analizzare le esatte implementazioni, che sono presenti nel capitolo successivo.\\
Il terzo capitolo è interamente dedicato al progetto. Infatti, offre un'analisi approfondita del contesto di lavoro, dell'attuazione pratica delle tecnologie precedentemente esaminate e dell'interazione interna, tra i membri del gruppo, ed esterna, tra le componenti front-end e back-end. Viene quindi illustrata la creazione del database e le procedure di persistenza dei dati, evidenziando le scelte relative a quali dati mantenere in memoria e le motivazioni sottostanti. Per finire, viene presentata la fase di test relativa al servizio back-end.\\
Il quarto capitolo costituisce la conclusione della tesi. Questa sezione riassume le parti principali esposte nell'elaborato e affronta questioni conclusive riguardanti il servizio realizzato, quali eventuali limitazioni dell’applicativo e possibili evoluzioni future attualmente non ancora implementate.



